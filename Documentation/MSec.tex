%#####################################################################################################################
% Datei	: MSec.tex
% Autor	: Byron Worms
%#####################################################################################################################
% Dokumenteinstellungen
\documentclass[conference]{acmsiggraph}

% Umlaute etc..
\usepackage[latin1]{inputenc}						
\usepackage[T1]{fontenc}

% Bibliothek f�r das Quellenverzeichnisse
\let\bibhang\relax
\usepackage[style=alphabetic,backend=biber,sorting=none]{biblatex}
\addbibresource{Literaturverzeichnis.bib}
\renewcommand{\bibfont}{\small}

% Bilder unterst�tzen
\usepackage[]{graphicx}

% PDF Inkludierung unterst�tzen
\usepackage{pdfpages}

% Farben unterst�tzen
\usepackage{xcolor}
\usepackage{color}
\usepackage{colortbl}

% Verweise auf Title
\usepackage{titleref}

% Sprachunterst�tzung (Letzte Sprache wird als Richtlinie genutzt!)
\usepackage[english,ngerman]{babel}
\usepackage{csquotes}

% Symbolunterst�tzung wie €
\usepackage{textcomp}

% Anhang unterst�tzen
\usepackage[titletoc]{appendix}

% Erweiterte Fu�noten unterst�tzen
\usepackage[hang]{footmisc}
\setlength{\footnotemargin}{1em}

% Unterst�tzung mehrzeiliger Kommentare
\usepackage{comment}

% Erweiterte Tabellen unterst�tzen
\usepackage{booktabs}
\usepackage{tabularx}
\usepackage{tabulary}
\usepackage{longtable,tabu}
\usepackage{array}

% Mathematische Operationen in {} Ausdr�cken erlauben (z.B. {\linewidth-4cm})
\usepackage{calc}

% "float" Positionierung unterst�tzen
\usepackage{scrhack}
\usepackage{float}

% Erweiterte Aufz�hlungen unterst�tzen
\usepackage{enumitem}

% Erweiterte Tabellenlinien unterst�tzen
\usepackage{hhline,float}

% Rotationen unterst�tzen
\usepackage{rotating}

% Erweiterte Symbole unterst�tzen
\usepackage{bbding} 
\usepackage{dingbat}
\usepackage{wasysym}
\usepackage{amssymb}

% Begrenzungsrahmen anzeigen lassen
%\usepackage{showframe}

% Tabellen werden nicht von Latex neupositioniert
\restylefloat{table}

% Bilder werden nicht von Latex neupositioniert
\restylefloat{figure}

% PDF Version 1.6 erzwingen
\pdfminorversion=6

% Kompression setzen (hyperref setzt das Level auf 9!)
\pdfcompresslevel=9

% Verschiebungen und Formattierungen des Modus "twoside" aufheben
\raggedbottom

% Diverse Hilfsfunktionen
\newcommand{\boldT}{\normalfont \bfseries}
\newcommand{\boldtext}[1]{{\normalfont \bfseries {#1}}}

% Eigene Definition von "\toprule" und "\bottomrule".
% Verhindert den Abstand zwischen den beiden Linien und den Vertikalen
\newcommand{\mytoprule}{\specialrule{1.25pt}{0em}{0em}}
\newcommand{\mybottomrule}{\specialrule{1.25pt}{0em}{0em}}
\newcommand{\mytoprulehalf}{\specialrule{0.75pt}{0em}{0em}}
\newcommand{\mybottomrulehalf}{\specialrule{0.75pt}{0em}{0em}}

% Eigene Definition f�r das Einbinden von Grafiken (erzeugt automatisch einen Rahmen um die Grafik)
% Bsp: \framepicture[width=0.5\textwidth, page=1]{...}
\newcommand{\framepicture}[2][\empty]{\fbox{\includegraphics[#1]{#2}}}

% Definierte Farben
\definecolor{mygreen}{rgb}{0.7,0.9,0.11}
\definecolor{myorange}{rgb}{1.0,0.49,0.15}

% Titel und Autoren festlegen
\title{SS15 Topic 05: Perceptual Hashing of Images}
\author{
	Byron Worms\\
	Multimedia and Security\\
	Fakult�t f�r Informatik\\
	byron.worms@st.ovgu.de
	}
\pdfauthor{Byron Worms}

% Schlagw�rter festlegen
\keywords{multimedia, security, hashing, perceptual, images, hash, robust hash, fingerprint}

%---------------------------------------------------------------------------------------------------------------------
% Einstiegspunkt
%---------------------------------------------------------------------------------------------------------------------
\begin{document}
	\maketitle						% Title generieren
	%#####################################################################################################################
% Datei	: Abstract.tex
% Autor	: Byron Worms
%#####################################################################################################################
\section*{Kurzfassung}
ToDo
			% Abstrakt einbinden	
	\keywordlist					% Keywords ausgeben
	\copyrightspace					% Copyright-Sektion generieren
	%................................................................
	% Ab hier den eigentlichen Inhalt einf�gen
	%#####################################################################################################################
% Datei	: Introduction.tex
% Autor	: Byron Worms
%#####################################################################################################################
%---------------------------------------------------------------------------------------------------------------------
% Motivation
%---------------------------------------------------------------------------------------------------------------------
\section{Motivation}
\label{sec:motivation}
ToDo

%---------------------------------------------------------------------------------------------------------------------
% Description of the used approach
%---------------------------------------------------------------------------------------------------------------------
\section{Perceptual Hashing of Images}
ToDo
		% Einleitung einbinden
	%#####################################################################################################################
% Datei	: TaskDescription.tex
% Autor	: Byron Worms
%#####################################################################################################################
%---------------------------------------------------------------------------------------------------------------------
% Problem description and investigation tasks
%---------------------------------------------------------------------------------------------------------------------
\section{Problembeschreibung}
\label{sec:tasks}
F�r eine aussagekr�ftige Analyse der berechneten �hnlichkeitsma�e und deren Bewertung ist ein breit aufgestelltes
Spektrum an verschiedenen Testdaten erforderlich. Der Umfang des zu testenden Bildmaterials betr�gt insgesamt 297 variierende
�hnliche sowie nicht �hnliche Bilder, das in einer Gesamtanzahl von 12501 Vergleichskombinationen (unter
Ber�cksichtigung der originalen Gruppierung, Abschnitt~\ref{sec:solution_testdata} \textit{Testdaten}) resultiert.
\\
Das definierte Bildmaterial dient als Eingabewert von vier unterschiedlichen Algorithmen f�r Perceptual Hashing �ber den
Bildraum, die in der C++ Bibliothek \textit{pHash} implementiert sind (~\cite{PHASH}):
\begin{itemize}[noitemsep]
	\item Radial hash projections (RADISH)
	\item Discrete cosine transform (DCT)
	\item Marr/mexican hat wavelet (wavelet)
	\item Block mean value based (BMB)
\end{itemize}
\noindent
Die resultierende Auswertung der durchgef�hrten Analysen beschreibt den n�heren Zusammenhang zwischen den berechneten
�hnlichkeitsma�en bezugnehmend auf den festgesetzten Threshold sowie die vier verwendeten Algorithmen.
\\
Zus�tzlich erfolgt eine experimentelle Untersuchung der potentiellen Geschwindigkeitssteigerung bei der Verwendung von
Perceptual Hashing gegen�ber dem Vergleich von Bilddaten mit vollst�ndigen, nicht auf aussagekr�ftigen Merkmale reduzierten
Informationss�tzen.
\\
Die Ausf�hrung der definierten Experimente und Analysen sowie die Extraktion und Berechnung von
relevanten Datenmengen basiert ausschlie�lich auf der Bibliothek \textit{pHash} und der daf�r eigens konzipierten und
entwickelten Benutzeroberfl�che. Eine b�ndige Erl�uterung der Oberfl�che kann dem Anhang~\ref{sec:appendix_b}
\textit{Dokumentation der verwendeten Daten} entnommen werden.
	% Aufgabenstellung einbinden
	%#####################################################################################################################
% Datei	: Solution.tex
% Autor	: Byron Worms
%#####################################################################################################################
%---------------------------------------------------------------------------------------------------------------------
% Solution approach
%---------------------------------------------------------------------------------------------------------------------
\section{L�sungsansatz}
\label{sec:solution}
F�r das Verst�ndnis der sowohl grundlegenden Funktionsweisen der verschiedenen Algorithmen als auch die Wahl des
variierenden Bildmaterials, ist ein Basiswissen �ber den allgemeinen Aufbau von Bildern und deren digitale Repr�sentation 
erforderlich.
\\
In Abh�ngigkeit des gew�hlten Farbmodells werden Bilder oftmals durch eine Koeffizientenmatrix von Helligkeitswerten
beschrieben. Im Beispiel des RGB--Farbmodells ist die finale Pixelfarbe durch drei unterschiedliche Koeffizienten der
einzelnen Farbkan�le (rot, gr�n und blau) definiert. Ein Graustufenbild besitzt dahingegen nur einen Farbkanal.
\\
Die Definition[SYN] und Belegung einer gegebenen Matrix aus Koeffizienten erzeugt dabei ein digitales Bild und/oder Muster, 
welches strukturell in hochfrequente als auch in niederfrequente Bereiche unterteilt ist. Gleichm��ige Unterschiede innerhalb
der Bildstruktur werden durch die niederfrequenten Anteile abgebildet, wohingegen hochfrequente Anteile[SYN] mit ungleichm��igen
sowie pl�tzlichen strukturellen �nderungen zum Detail des Gesamtbilds beitragen.

\subsection{Testdaten}
\label{sec:solution_testdata}
Das verwendete Bildmaterial ist in die drei unten aufgef�hrten Kategorien eingeteilt. Die Unterteilung des Gesamtdatensatzes
dient einer Komplexit�t abh�ngigen, verbesserten Erkenntnisgewinnung der jeweiligen Arbeitsweisen der verschiedenen eingesetzten
Algorithmen. Somit soll ein Zusammenspiel zwischen den Fehlfunktionen[SYN] der Algorithmen und dem strukturellen Aufbau der
Bilder leichter detektiert werden.

\subsubsection*{Einfache Farben}

\subsubsection*{Elementar Formen}

\subsubsection*{Komplexe Bilder}

\subsection{Algorithmen}
\label{sec:solution_algorithms}

\subsubsection*{RADISH}

\subsubsection*{DCT}

\subsubsection*{Wavelet}

\subsubsection*{BMB}


\begin{itemize}
	\item Struktur eines Bilder (hochfrequent, niedrigfrequent)
	\item Bilder (Einteilung in die Gruppen (Feature-Gruppen), wieso macht man das, Manipulation -> Anhang)
	\item Vier Algorithmen (Gemeinsamkeiten, Tabelle mit Kerneigenschaften)
	\item Kurz die unterschiedlichen Algorithmen erl�utern
	\item http://www.debugmode.com/imagecmp/classify.htm	
\end{itemize}			% L�sungsansatz einbinden
	%#####################################################################################################################
% Datei	: Experimental.tex
% Autor	: Byron Worms
%#####################################################################################################################
%---------------------------------------------------------------------------------------------------------------------
% Experimental Investigation
%---------------------------------------------------------------------------------------------------------------------
\section{Experimentelle Untersuchungen}
\label{sec:experimental}
ToDo		% Experimentelle Untersuchungen einbinden
	%#####################################################################################################################
% Datei	: Results.tex
% Autor	: Byron Worms
%#####################################################################################################################
%---------------------------------------------------------------------------------------------------------------------
% Test results
%---------------------------------------------------------------------------------------------------------------------
\section{Testergebnisse}
\label{sec:results}
\begin{itemize}
	\item Determinimus 
\end{itemize}			% Ergebnisse einbinden
	%#####################################################################################################################
% Datei	: Conclusion.tex
% Autor	: Byron Worms
%#####################################################################################################################
%---------------------------------------------------------------------------------------------------------------------
% Future work
%---------------------------------------------------------------------------------------------------------------------
\section{Abschluss}
\label{sec:conclusion}
In dem vorliegenden technischen Paper wurden vier verschiedene Algorithmen aus dem Bereich \textit{Perceptual Hashing}
�ber den Bildraum mit Hilfe mehrerer verschieden definierten Experimenten analysiert und ausgewertet. Die Ergebnisse
der Untersuchungen werden nachfolgend zusammengefasst aufgef�hrt.
\\[1em]
Jeder eingesetzte Algorithmus arbeitet auf Basis des strukturellen Aufbau des Bildmaterials (eingeteilt in hochfrequente
und niederfrequente Anteile). Die Komplexit�t der Struktur hat dabei einen erheblichen Einfluss auf die Zuordnungsqualit�t
w�hrend der Bestimmung von �hnlichkeitsma�en. Die Bedeutsamkeit der Bildmodifikationen steigt parallel mit der Erh�hung
der strukturellen Transparenz des Bildmaterials, sodass bereits leichte Ver�nderungen bei einfachen Bildinhalten zu gro�en
�nderungen im Aufbau f�hren.
\\
Besonders der \textit{Wavelet}--Algorithmus ist durch die Adressierung von hochfrequenten Bildbereichen (vgl. grundlegende
Funktionsweisen von grafischen Kantendetektierungen) anf�llig gegen�ber geringf�gigen Ver�nderungen. Dahingegen sind
Rotations�nderungen (zum Beispiel horizontales Spiegel) eine Schwachstelle, die alle der vier verwendeten Algorithmen betrifft.
\\
Die Auswertung des durchgef�hrten Quervergleichs der Bildkategorie \textit{Komplexe Bilder} f�hrte zu dem Ergebnis,
dass die \textit{RADISH}--Vorgehensweise durch �hnliche Intensit�tsverteilungen in den abgeglichenen Bildmaterial zu einer
falschen Zuordnung von �hnlichen Bildpaaren neigt. Die Erkennungsqualit�ten des \textit{BMB}--Verfahrens sinken dahingegen
mit der gleichzeitigen Verringerungen des Thresholds.
\\[1em]
Ungeachtet der gemessenen Erkennungsfehlerraten operieren die Algorithmen \textit{BMB} und \textit{RADISH} durch die
ausschlie�liche Nutzung einfacher Pixeloperationen effizienter als die Verfahren \textit{DCT} sowie \textit{Wavelet},
die kostenintensive Operationen auf Pixelebene einsetzen.
\\
In einem direkten Vergleich der vier Algorithmen zu der nativen Herangehensweise mittels \textit{Kreuzkorrelation} wurde
eine durchschnittliche Geschwindigkeitssteigerung von bis zu \hbox{$\sim10500\%$} protokolliert.
\\[1em]
Auf Basis der gemessenen Erkennungsraten sowie Berechnungszeiten besteht die M�glichkeit, ein \textit{Ranking} der vier
Algorithmen zu Gunsten der nachstehenden Bewertungsgruppen zu definieren: \textit{FRR}, \textit{FAR}, \textit{Berechnungszeiten}
und die Gesamtwertung.
\\
Die einzelnen Fehlerraten und Zeiten werden dabei aufaddiert und anschlie�end durch die kumulierte obere Grenze dividiert.
Die vereinfachte Rechnung ist anwendbar, da alle quantifizierten Messwerte in dem Bereich von \hbox{$[0,1]$} liegen. Trotzdem
ist die nachstehende \textit{Ranking}--Tabelle~\ref{tab:conclusion_ranking} nur ein Sch�tzma� f�r die allgemeine Qualit�t und
Effizienz der Algorithmen. Zum einem werden nicht vereinbare Einheiten miteinander in Verbindung gesetzt (Fehlerraten und Sekunden)
und zum anderen sind die in Abschnitt~\ref{sec:experimental} ermittelten Ma�e abh�ngig von dem ausgew�hlten Bildmaterial.
Ein Austausch des eingesetzten Materials kann dementsprechend zu einer differenzierenden \textit{Ranking}--Tabelle f�hren.
\begin{table}[H]
	\begin{center}
		\begin{tabular}{|l|c|c|c|c|}
			\mytoprule
			\centering\bfseries Eigenschaft & \bfseries \textit{RADISH} & \bfseries \textit{DCT} & \bfseries \textit{Wavelet} & \bfseries \textit{BMB}
			\\
			\hline
			\hline
			%---------------------------------------------------------------------------------------------------------------------	
			\textit{FRR} & 0,39 & 0,469 & \cellcolor{myorange}0,828 & \cellcolor{mygreen}0,256
			\\
			\hline
			%---------------------------------------------------------------------------------------------------------------------	
			\textit{FAR} & \cellcolor{myorange}0,028 & 0,0009 & \cellcolor{mygreen}0 & 0,02
			\\
			\hline	
			%---------------------------------------------------------------------------------------------------------------------	
			\textit{Geschwindigkeit} & 0,107s & 0,223s & \cellcolor{myorange}0,362s & \cellcolor{mygreen}0,038s 
			\\	
			\hline
			\hline
			%---------------------------------------------------------------------------------------------------------------------	
			\textit{Gesamt} & 0,175 & 0,23 & \cellcolor{myorange}0,39 & \cellcolor{mygreen}0,104 
			\\																						
			%---------------------------------------------------------------------------------------------------------------------
			\mybottomrule			
		\end{tabular}
		\caption{Ranking der Algorithmen (Eigene Darstellung)}
		\label{tab:conclusion_ranking}
	\end{center}
\end{table}
\noindent
Der Algorithmus mit den besten Werten bez�glich der betrachteten Eigenschaft wird mit der Farbe {\color{mygreen}\rule[0cm]{0.2cm}{0.2cm}}
hervorgehoben, wohingegen das Verfahren mit den schlechtesten Werten mit der Farbe {\color{myorange}\rule[0cm]{0.2cm}{0.2cm}} kodiert
wird.

\subsection{Zusammenfassung}
\label{sec:summary}
ToDo

\subsection{Ausblick}
\label{subsec:futureWork}
ToDo		% Zusammenfassung/Fazit einbinden	
	%................................................................
	\sloppy
	\newpage
	\printbibliography				% Literaturverzeichnis ausgeben
	
	\onecolumn						% Der Anhang erfolgt einspaltig!
	\begin{appendices}
		%#####################################################################################################################
% Datei	: AppendixA.tex
% Autor	: Byron Worms
%#####################################################################################################################
\section{Aufgabenbeschreibung}
\label{sec:appendix_a}
\boldtext{Tasks:}
\begin{comment}
\begin{enumerate}
	\item[\bold (a)] \boldtext{Realization of an own setup of demonstrators}\\
	The task is to implement a keystroke dynamics (KD) acquisition and matching prototype from scratch (in JAVA).\\
	The tasks of the group are to implement an own demonstrator for this simple to acquire and process modality,
	using di-graph [Sim2007] and tri-graph representations of fixed-length input as feature space and use this
	prototype to perform user authentication.
	%----------------------------------------------------------------------------------------------------------------
	\item[\bold (b)] \boldtext{Data to be used}\\
	This group has to collect with their own tool KD data of the course participants. The set of samples must
	contain the following semantics:
	\begin{itemize}[noitemsep]
		\item[-] A pseudonym
		\item[-] A given pin ("77993")
		\item[-] A pin chosen by the donor of the samples
		\item[-] A sketch of a symbol chosen by the donor of the samples
		\item[-] The answer to the question "Where are you from?"
	\end{itemize}
	Furthermore, the participants in this group are asked to contribute to the data collections of the handwriting (HW)
	and (other) keystroke dynamic (KD) groups!
	%----------------------------------------------------------------------------------------------------------------	
	\item[\bold (c)] \boldtext{Evaluation}\\
	Run your prototype (see subtask (a) above) on the collected data (see subtask (b) above) and perform a performance
	evaluation with your prototype. The evaluation must include (but is not limited to):
	\begin{itemize}[noitemsep]
		\item An analysis of the authentication performance of the system using your own collected data
		\item An attack attempt using blind imposter attacks trying to force errors
		\item A projection of the samples in your data set to the characters of 'Doddingtons Zoo' [Doddington1998] and
		- if possible - an application of 'Doddingtons rules of thumbs' for the evaluation of the authentication performance
		of your demonstrator and data set
		\item A summary of the overall approach/algorithms/results/impact of the individual topic and a projection onto
		the biometric processing pipeline (see e.g. [Vielhauer2006])
	\end{itemize}
	%----------------------------------------------------------------------------------------------------------------	
	\item[\bold (d)] \boldtext{Presentation}\\
	Documentation of the work performed is done in a (structured) layout (ACM layout with max. 8 pages plus appendix)
	and presentation in lecture time (2x10 minutes presentation time for two-student team or 15 minutes for a one-student
	team).	
\end{enumerate}
\end{comment}	% Anhang A einbinden (Aufgabebeschreibung)
		%#####################################################################################################################
% Datei	: AppendixB.tex
% Autor	: Byron Worms
%#####################################################################################################################
\section{Dokumentation der verwendeten Daten}
\label{sec:appendix_b}
Wie in Abschnitt~\ref{sec:solution} (\textit{L�sungsansatz}) beschrieben, wurde das Bildmaterial in unterschiedliche
Datens�tze aufgeteilt. Dabei sind die ersten beiden Datens�tze[SYN] mittels des von Windows angebotenen Programms
\textit{Microsoft Paint} erstellt worden, w�hrend der dritte Datensatz mit Hilfe von diversen Digitalkameras generiert
wurde.
\\
Die auf den Datens�tzen angewandten Modifikationen und �nderungen sind mit dem Programm \textit{GIMP} durchgef�hrt worden.
Art und Weise sowie die Parametrisierung der Ab�nderungen k�nnen der nachstehenden Auflistung entnommen werden:
\begin{description}[noitemsep,nolistsep]
	\item[{\normalfont 2. Datensatz:}] \hfill
	\begin{itemize}[noitemsep,nolistsep]
		\item Skalierung: Faktoren = $\frac{1}{6}x, 6x$, Interpolation = kubisch
		\item Gammakorrektur: Gamma = 9
		\item Rauschen: Typ = \textit{Hurl}, Randomisierung = $80\%$, Wiederholung = 1
	\end{itemize}
	\item[{\normalfont 3. Datensatz:}] \hfill
		\begin{itemize}[noitemsep,nolistsep]
			\item Skalierung: Faktoren = $\frac{1}{6}x, 2x$, Interpolation = kubisch
			\item Gammakorrektur: Untere Grenze: 128, Gamma = 4
			\item Rauschen: Typ = \textit{Hurl}, Randomisierung = $60\%$, Wiederholung = 1
		\end{itemize}
\end{description}
\noindent
Alle Durchf�hrungsschritte der Analysen erfolgten mit dem eigens konzipierten und anschlie�end erstellten Programms
\textit{Perceptual Image Hashing}. Eine kurze und b�ndige Erl�uterung der grafischen Oberfl�che kann den nachfolgenden
Sektionen~\ref{subsec:appendix_b_imgvsimg} und~\ref{subsec:appendix_b_cc} entnommen werden.
\\
Folgende Bildformate werden f�r den Import von Bildmaterial unterst�tzt: \textit{JPEG, Bitmap, PNG}.
\newpage

\subsection{Grafische Oberfl�che: Image vs. Image}
\label{subsec:appendix_b_imgvsimg}
Die nachfolgende Abbildung~\ref{fig:appendix_b_imgvsimg} repr�sentiert die grafische Benutzeroberfl�che f�r den
direkten Vergleich zweier Bilddateien miteinander. Der vom gew�hlten Algorithmus berechnete \textit{Hash} ist ebenfalls
ablesbar.
\begin{figure}[H]
	\centering
	\framepicture[page=1,width=0.96\linewidth]{Pictures/ImgVsImgExplained.pdf}
	\caption{Benutzeroberfl�che: Image vs. Image (Eigene Darstellung)}
	\label{fig:appendix_b_imgvsimg}
\end{figure}
\begin{enumerate}[noitemsep]
	\item Erm�glicht die Auswahl des zu nutzenden Algorithmus und dessen Parametrisierung (falls vorhanden).
	\item Bietet die Auswahl der zu vergleichenden Bilddateien an. F�r eine bessere Wiedererkennung werden die ausgew�hlten
	Bilder durch eine Vorschau grafisch dargestellt. Nach erfolgreicher Berechnung der \textit{Hash}--Daten k�nnen diese
	in dem unteren Bereich abgelesen werden.
	\item Startet die Berechnung der \textit{Hash}--Daten und deren anschlie�enden Vergleich. Die Operation ist nur verf�gbar,
	falls beide Slots f�r die Bilddateien belegt wurden.
	\item Gibt die vom gew�hlten Schwellenwert abh�ngige errechnete �hnlichkeit f�r das Bildpaar an. F�r eine verbesserte Darstellung
	wird das Ergebnis abh�ngig von seiner Akzeptanz entweder gr�nlich (�hnlichkeitsma� liegt �ber dem Schwellenwert) oder
	r�tlich (�hnlichkeit liegt unterhalb des Schwellenwerts) angezeigt.
\end{enumerate}

\subsection{Grafische Oberfl�che: Cross Comparison}
\label{subsec:appendix_b_cc}
\begin{figure}[H]
	\centering
	\framepicture[page=1,width=0.96\linewidth]{Pictures/CCExplained.pdf}
	\caption{Benutzeroberfl�che: Cross Comparison (Eigene Darstellung)}
	\label{fig:appendix_b_cc}
\end{figure}
\begin{enumerate}[noitemsep]
	\item �hnlich zu dem in Sektion~\ref{subsec:appendix_b_imgvsimg} (Grafische Oberfl�che: Image vs. Image) beschriebenen Verhalten.
	Eine Erweiterung stellt die m�gliche Mehrfachauswahl der Techniken/Algorithmen, die w�hrend der Berechnung verwendet werden sollen,
	dar.
	\item In dem Vergleichsverfahren werden keine einzelnen Bilder f�r die Gegen�berstellung ausgew�hlt, sondern ein Quellverzeichnis,
	welches das zu untersuchende Bildmaterial beinhaltet. Die g�ltigen Bilddateien werden automatisch extrahiert und f�r den Vergleich
	gez�hlt sowie vorbereitet. Die Anzahl der in Frage kommenden Bildern wird seitlich angezeigt.
	\item Bietet eine Ansammlung von unterschiedlichen Operationen an (links beginnend):
	\begin{itemize}
		\item Startet den Vergleichsprozess. Je nach Anzahl und Komplexit�t des Bildmaterials sowie der ausgew�hlten Algorithmen
		kann dies einige Minuten dauern.
		\item Gruppen werden mit nur einem Klick alle auf einmal geschlossen oder ge�ffnet (Vergleich Punkt 5.). Diese Funktion ist erst
		nach erfolgreicher Berechnung f�r den Einsatz verf�gbar.
		\item Selektiert alle in der Liste (Vergleich Punkt 4.) vorhanden Vergleichspaare.
		\item Liefert eine �bersicht der durchschnittlichen Zeitwerte f�r die einzelnen Algorithmen hinsichtlich derer Lade-- und
		Berechnungszeiten.
	\end{itemize}
	\item Beinhaltet eine Auflistung aller ausgew�hlten Vergleichspaare und deren Ma� an �hnlichkeit zu einander in Abh�ngigkeit
	des betrachteten Verfahrens. F�r eine verbesserte grafische Darstellung werden die einzelnen Werte farblich hervorgehoben. Der Basisgr�nton
	wird f�r alle Elemente mit einer �hnlichkeit gleich oder oberhalb des gesetzten Schwellenwerts verwendet. Eine lineare Interpolation zwischen
	gr�n ($\ddot{U}bereinstimmung \stackrel{\wedge}{=} Schwellenwert$) und rot ($\ddot{U}bereinstimmung \stackrel{\wedge}{=} 0$) wird f�r
	die �brigen Elemente durchgef�hrt.
	\item Die in dem Vergleichsverfahren \textit{Cross Comparison} entstehende Anzahl von Ergebnissen ist bereits bei einer kleinen Menge an
	verwendeten Bilddateien umfangreich und schnell un�bersichtlich. Viele der erzeugte Paare sind f�r eine weitere Auswertung oftmals
	�berfl�ssig und k�nnen somit der Anzeige entfallen. Mit Hilfe der Filterbox k�nnen entweder bereits vordefinierte und/oder selbst erstellte
	Filter auf die Gesamtergebnismenge f�r eine Reduzierung des Datenbestandes angewandt werden. Die Paare k�nnen zur F�rderung der 
	�bersichtlichkeit in unterschiedliche Gruppen eingeteilt werden. Der vollst�ndige Befehlssatz und deren Syntax kann der Software
	beiliegenden Datei \textit{Filter.txt} entnommen werden.
	\item Bei der Selektion eines einzelnen Paaren werden Informationen mit Hilfe eines ausklappenden Fensters angezeigt. Gegebenenfalls besteht
	die M�glichkeit die einzelnen Abarbeitungsschritte der verschiedenen Algorithmen anzeigen zu lassen.
\end{enumerate}	% Anhang B einbinden (Dokumentation der benutzen Daten)
		%#####################################################################################################################
% Datei	: AppendixC.tex
% Autor	: Byron Worms
%#####################################################################################################################
\newpage
\section{Aufgetretene Probleme und offene Punkte}
\label{sec:appendix_c}
W�hrend der Durchf�hrung der verschiedenen Analysen und Experimenten konnten unterschiedliche Probleme festgestellt
werden:
\begin{itemize}
	\item Bei dem Versuch ein Bild in dem Dateiformat \textit{PNG} zu importieren, st�rzte die f�r die Aufgabe vorgegebene
	externe Bibliothek \textit{phash} mit unbekanntem Fehler ab. Die Ursache liegt in einer von \textit{phash} verwendeten
	Bibliothek f�r das Laden von Bildmaterialien aus unterschiedlichen Dateiformaten \textit{CImg}.
	\item Bei dem dritten Datensatz f�hrten Bildaufl�sungen gr��er als $2048xBel.$ zu nicht deterministischen Abst�rzen der
	externen Bibliothek \textit{phash}. Ursachen f�r diesen Fehler sind unbekannt.
\end{itemize}	% Anhang C einbinden (Problembeschreibungen und offene Punkte)
		%#####################################################################################################################
% Datei	: AppendixD.tex
% Autor	: Byron Worms
%#####################################################################################################################
\section{Pr�sentationsfolien}
\label{sec:appendix_d}
ToDo	% Anhang D einbinden (Pr�sentationsfolien)
		%#####################################################################################################################
% Datei	: AppendixE.tex
% Autor	: Byron Worms
%#####################################################################################################################
\label{sec:appendix_e}
\includepdf[pages={1},scale=.85,pagecommand=\section{Discussion Tables}]{Pictures/table_watermarking.pdf}
\includepdf[pages={1}]{Pictures/table_digital_forensics.pdf}
\includepdf[pages={1}]{Pictures/table_perceptual_hasing.pdf}
\includepdf[pages={1}]{Pictures/table_stego.pdf}	% Anhang E einbinden (Discussion Tables)
	\end{appendices}
\end{document}