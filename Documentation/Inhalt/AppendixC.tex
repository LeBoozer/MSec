%#####################################################################################################################
% Datei	: AppendixC.tex
% Autor	: Byron Worms
%#####################################################################################################################
\newpage
\section{Aufgetretene Probleme und offene Punkte}
\label{sec:appendix_c}
W�hrend der Durchf�hrung der verschiedenen Analysen und Experimenten konnten unterschiedliche Probleme festgestellt
werden:
\begin{itemize}
	\item Bei dem Versuch ein Bild in dem Dateiformat \textit{PNG} zu importieren, st�rzte die f�r die Aufgabe vorgegebene
	externe Bibliothek \textit{pHash} mit unbekanntem Fehler ab. Die Ursache liegt in einer von \textit{pHash} verwendeten
	Bibliothek f�r das Laden von Bildmaterialien aus unterschiedlichen Dateiformaten \textit{CImg}.
	\item Bei dem dritten Datensatz f�hrten Bildaufl�sungen gr��er als $2048xBel.$ zu nicht deterministischen Abst�rzen der
	externen Bibliothek \textit{pHash}. Ursachen f�r diesen Fehler sind unbekannt.
	\item Die Nutzung der Daten aus dem \textit{Course media repository} (CMR) resultierte, �hnlich zu dem voranstehenden
	Punkt in nicht nachvollziehbare Abst�rze der externen Bibliothek \textit{pHash} oder der von \textit{pHash} verwendeten
	Bibliothek \textit{CImg}. Aufgrund der teils immensen Reduzierung des nutzbaren Bildmaterials aus dem CMR, wurden
	die restlichen Bilder der potentiellen CMR--Inhalte nicht weiter analysiert.
	\item Mehrere CMR--Inhalte waren f�r eine eingehenden Analyse mittels \textit{Perceptual Hashing} entweder durch die oben
	genannten Probleme oder durch einen stark vereinfachten Bildumfang nicht brauchbar. Diese Daten wurden ebenfalls nicht
	n�her analysiert.
\end{itemize}