%#####################################################################################################################
% Datei	: Solution.tex
% Autor	: Byron Worms
%#####################################################################################################################
%---------------------------------------------------------------------------------------------------------------------
% Solution approach
%---------------------------------------------------------------------------------------------------------------------
\section{L�sungsansatz}
\label{sec:solution}
F�r das Verst�ndnis der sowohl grundlegenden Funktionsweisen der verschiedenen Algorithmen als auch die Wahl des
variierenden Bildmaterials, ist ein Basiswissen �ber den allgemeinen Aufbau von Bildern und deren digitale Repr�sentation 
erforderlich.
\\
In Abh�ngigkeit des gew�hlten Farbmodells werden Bilder oftmals durch eine Koeffizientenmatrix von Helligkeitswerten
beschrieben. Im Beispiel des RGB--Farbmodells ist die finale Pixelfarbe durch drei unterschiedliche Koeffizienten der
einzelnen Farbkan�le (rot, gr�n und blau) definiert. Ein Graustufenbild besitzt dahingegen nur einen Farbkanal.
\\
Die Definition[SYN] und Belegung einer gegebenen Matrix aus Koeffizienten erzeugt dabei ein digitales Bild und/oder Muster, 
welches strukturell in hochfrequente als auch in niederfrequente Bereiche unterteilt ist. Gleichm��ige Unterschiede innerhalb
der Bildstruktur werden durch die niederfrequenten Anteile abgebildet, wohingegen hochfrequente Anteile[SYN] mit ungleichm��igen
sowie pl�tzlichen strukturellen �nderungen zum Detail des Gesamtbilds beitragen.

\subsection{Testdaten}
\label{sec:solution_testdata}
Das verwendete Bildmaterial ist in die drei unten aufgef�hrten Kategorien eingeteilt. Die Unterteilung des Gesamtdatensatzes
dient einer Komplexit�t abh�ngigen, verbesserten Erkenntnisgewinnung der jeweiligen Arbeitsweisen der verschiedenen eingesetzten
Algorithmen. Somit soll ein Zusammenspiel zwischen den Fehlfunktionen[SYN] der Algorithmen und dem strukturellen Aufbau der
Bilder leichter detektiert werden.

\subsubsection*{Einfache Farben}

\subsubsection*{Elementar Formen}

\subsubsection*{Komplexe Bilder}

\subsection{Algorithmen}
\label{sec:solution_algorithms}

\subsubsection*{RADISH}

\subsubsection*{DCT}

\subsubsection*{Wavelet}

\subsubsection*{BMB}


\begin{itemize}
	\item Struktur eines Bilder (hochfrequent, niedrigfrequent)
	\item Bilder (Einteilung in die Gruppen (Feature-Gruppen), wieso macht man das, Manipulation -> Anhang)
	\item Vier Algorithmen (Gemeinsamkeiten, Tabelle mit Kerneigenschaften)
	\item Kurz die unterschiedlichen Algorithmen erl�utern
	\item http://www.debugmode.com/imagecmp/classify.htm	
\end{itemize}