%#####################################################################################################################
% Datei	: TaskDescription.tex
% Autor	: Byron Worms
%#####################################################################################################################
%---------------------------------------------------------------------------------------------------------------------
% Problem description and investigation tasks
%---------------------------------------------------------------------------------------------------------------------
\section{Problembeschreibung}
\label{sec:tasks}
F�r eine aussagekr�ftige Analyse der berechneten �hnlichkeitsma�e und deren Bewertung ist ein breit aufgestelltes
Spektrum an verschiedenen Testdaten erforderlich. Der Umfang des zu testenden Bildmaterials betr�gt insgesamt 297 variierende
�hnliche sowie nicht �hnliche Bilder, das in einer Gesamtanzahl von 12501 Vergleichskombinationen (unter
Ber�cksichtigung der originalen Gruppierung, Abschnitt~\ref{sec:solution_testdata} \textit{Testdaten}) resultiert.
\\
Das definierte Bildmaterial dient als Eingabewert von vier unterschiedlichen Algorithmen f�r Perceptual Hashing �ber den
Bildraum, die in der C++ Bibliothek \textit{pHash} implementiert sind (\cite{PHASH}):
\begin{itemize}[noitemsep]
	\item Radial Hash Projections (RADISH)
	\item Discrete Cosine Transform (DCT)
	\item Marr/mexican hat Wavelet (wavelet)
	\item Block Mean Value Based (BMB)
\end{itemize}
\noindent
Die resultierende Auswertung der durchgef�hrten Analysen beschreibt den n�heren Zusammenhang zwischen den berechneten
�hnlichkeitsma�en bezugnehmend auf den festgesetzten Threshold sowie die vier verwendeten Algorithmen.
\\
Zus�tzlich erfolgt eine experimentelle Untersuchung der potentiellen Geschwindigkeitssteigerung bei der Verwendung von
Perceptual Hashing gegen�ber dem Vergleich von Bilddaten mit vollst�ndigen, nicht auf aussagekr�ftigen Merkmale reduzierten
Informationss�tzen.
\\
Die Ausf�hrung der definierten Experimente und Analysen sowie die Extraktion und Berechnung von
relevanten Datenmengen basiert ausschlie�lich auf der Bibliothek \textit{pHash} und der daf�r eigens konzipierten und
entwickelten Benutzeroberfl�che. Eine b�ndige Erl�uterung der Oberfl�che kann dem Anhang~\ref{sec:appendix_b}
(\textit{Dokumentation der verwendeten Daten}) entnommen werden.
