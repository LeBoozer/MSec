%#####################################################################################################################
% Datei	: Introduction.tex
% Autor	: Byron Worms
%#####################################################################################################################
%---------------------------------------------------------------------------------------------------------------------
% Motivation
%---------------------------------------------------------------------------------------------------------------------
\section{Motivation}
\label{sec:motivation}
Die bestehende und schnell anwachsende globale Vernetzung der Menschen untereinander f�rdert unter anderem den Austausch
sowie die Verbreitung von unterschiedlichsten Informationen �ber das Internet. Vor allem bei der Jugend beliebte Dienste
wie Facebook, Instagram oder Twitter dienen dabei oftmals als zugrundeliegende Plattform f�r den �ffentlichen Transfer
von verschiedenen Medientypen (z.B. Bilder, Videos, Audio und Text). Eine Vielzahl der bei den Diensten[SYN] oder auf
(privaten) Internetseiten eingesetzten Medien werden meist ohne Bedacht auf Urheberrechte (ver�ndert)[???] verwendet
und im schlimmsten Fall ohne Quellenangabe eingesetzt.
\\
Ein m�gliches Verfahren f�r die Zuordnung eines Urhebers zu dem von ihm erstellten Medium stellt das sogenannte
\textit{robuste Watermarking} dar. Hierbei werden digitale Signaturen direkt mit in das Mediaobjekt (nachfolgend f�r
Bilder, Videos und Audio) versteckt integriert, die f�r eine sp�tere Authentifizierung extrahiert werden k�nnen. Nachteile
des Watermarkings sind neben Performance intensiven Berechnungen unter anderem inhaltsbewahrende Ver�nderungen (z.B.
Format�nderungen), das Entfernen der Signaturen und die nicht bestehende M�glichkeit digitale Signaturen in bereits
publizierte Mediaobjekte nachtr�glich zu integrieren (Quelle: VL).
\\
Perceptual Hashing offeriert im Gegensatz dazu eine alternative Herangehensweise an die Problemstellung, welche auf die Performance
intensiven Integrationen von digitalen Signaturen verzichtet und auch inhaltsbewahrende Ver�nderungen mit ber�cksichtigt.

%---------------------------------------------------------------------------------------------------------------------
% Description of the used approach
%---------------------------------------------------------------------------------------------------------------------
\section{Perceptual Hashing of Images}
Mit dem Wort \textit{Hash} wird in der Regel ein kryptografisches Hashverfahren bezeichnet, das einen beliebig langen[SYN] Eingabewert,
unter Ber�cksichtigung des \textit{Avalance--Effekts}, auf einen meist festen Ausgabewert abbildet.
Diese Art des Hashings findet einen gro�en Einsatz in dem Bereich der Integrit�tspr�fung.
\\
Mediaobjekte k�nnen nach Ver�nderungen oder Manipulationen (z.B. Format�nderungen und Komprimierungen) f�r die menschlichen Sinne
�hnlich bis gleich wahrgenommen werden, unterscheiden sich in ihrer bin�ren Pr�sentation hingegen erheblich, sodass
der Einsatz[SYN] von kryptografischen Verfahren aufgrund des \textit{Avalance--Effekts} ungeeignet ist.
\\
Perceptual Hashing ber�cksichtigt dahingegen Ver�nderungen und Manipulationen in der zugrundeliegenden Arbeitsweise des Verfahrens[SYN], indem typische strukturelle Merkmale (im Vergleich zu detaillierten Merkmalen[SYN] stabiler gegen�ber Ver�nderungen) der Mediaobjekte extrahiert und miteinander verglichen werden. Je mehr Merkmale[SYN] miteinander
�bereinstimmen, desto h�her ist anschlie�end die Wahrscheinlichkeit, dass die beiden Mediaobjekte �hnlich bis gleich
zueinander sind. Das Ma� an Ber�cksichtigung von Korrekturen an dem Mediaobjekt ist abh�ngig von dem gew�hlten Algorithmus.
\\
Durch die Einf�hrung eines �hnlichkeitsma�es wird ein Threshold erforderlich, der in Bezug auf die errechnete �hnlichkeit Angaben zu der unteren Akzeptanzgrenze der Merkmals�bereinstimmungen macht. Im Gegensatz zu den kryptografischen Ans�tzen existieren bei dem Perceptual Hashing die Fehlerraten FAR (False--Acceptance--Rate) und FRR (False--Rejection--Rate)
[Irgendwas mit Threshold und nur Einbeziehung von �hnlichkeiten].
\\
In den nachfolgenden Abschnitten des vorliegenden Papers werden ausgew�hlte Algorithmen aus dem Bereich des Perceptual
Hashing von Bildern detailliert analysiert und deren Erkennungsraten mittels unterschiedlichen definierten Experimenten untersucht und bewertet.
