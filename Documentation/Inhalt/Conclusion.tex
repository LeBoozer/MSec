%#####################################################################################################################
% Datei	: Conclusion.tex
% Autor	: Byron Worms
%#####################################################################################################################
%---------------------------------------------------------------------------------------------------------------------
% Future work
%---------------------------------------------------------------------------------------------------------------------
\section{Abschluss}
\label{sec:conclusion}
In dem vorliegenden technischen Paper wurden vier verschiedene Algorithmen aus dem Bereich \textit{Perceptual Hashing}
�ber den Bildraum mit Hilfe mehrerer verschieden definierten Experimenten analysiert und ausgewertet. Die Ergebnisse
der Untersuchungen werden nachfolgend zusammengefasst aufgef�hrt.
\\[1em]
Jeder eingesetzte Algorithmus arbeitet auf Basis des strukturellen Aufbau des Bildmaterials (eingeteilt in hochfrequente
und niederfrequente Anteile). Die Komplexit�t der Struktur hat dabei einen erheblichen Einfluss auf die Zuordnungsqualit�t
w�hrend der Bestimmung von �hnlichkeitsma�en. Die Bedeutsamkeit der Bildmodifikationen steigt parallel mit der Erh�hung
der strukturellen Transparenz des Bildmaterials, sodass bereits leichte Ver�nderungen bei einfachen Bildinhalten zu gro�en
�nderungen im Aufbau f�hren.
\\
Besonders der \textit{Wavelet}--Algorithmus ist durch die Adressierung von hochfrequenten Bildbereichen (vgl. grundlegende
Funktionsweisen von grafischen Kantendetektierungen) anf�llig gegen�ber geringf�gigen Ver�nderungen. Dahingegen sind
Rotations�nderungen (zum Beispiel horizontales Spiegel) eine Schwachstelle, die alle der vier verwendeten Algorithmen betrifft.
\\
Die Auswertung des durchgef�hrten Quervergleichs der Bildkategorie \textit{Komplexe Bilder} f�hrte zu dem Ergebnis,
dass die \textit{RADISH}--Vorgehensweise durch �hnliche Intensit�tsverteilungen in den abgeglichenen Bildmaterial zu einer
falschen Zuordnung von �hnlichen Bildpaaren neigt. Die Erkennungsqualit�ten des \textit{BMB}--Verfahrens sinken dahingegen
mit der gleichzeitigen Verringerungen des Thresholds.
\\[1em]
Ungeachtet der gemessenen Erkennungsfehlerraten operieren die Algorithmen \textit{BMB} und \textit{RADISH} durch die
ausschlie�liche Nutzung einfacher Pixeloperationen effizienter als die Verfahren \textit{DCT} sowie \textit{Wavelet},
die kostenintensive Operationen auf Pixelebene einsetzen.
\\
In einem direkten Vergleich der vier Algorithmen zu der nativen Herangehensweise mittels \textit{Kreuzkorrelation} wurde
eine durchschnittliche Geschwindigkeitssteigerung von bis zu \hbox{$\sim10500\%$} protokolliert.
\\[1em]
Auf Basis der gemessenen Erkennungsraten sowie Berechnungszeiten besteht die M�glichkeit, ein \textit{Ranking} der vier
Algorithmen zu Gunsten der nachstehenden Bewertungsgruppen zu definieren: \textit{FRR}, \textit{FAR}, \textit{Berechnungszeiten}
und die Gesamtwertung.
\\
Die einzelnen Fehlerraten und Zeiten werden dabei aufaddiert und anschlie�end durch die kumulierte obere Grenze dividiert.
Die vereinfachte Rechnung ist anwendbar, da alle quantifizierten Messwerte in dem Bereich von \hbox{$[0,1]$} liegen. Trotzdem
ist die nachstehende \textit{Ranking}--Tabelle~\ref{tab:conclusion_ranking} nur ein Sch�tzma� f�r die allgemeine Qualit�t und
Effizienz der Algorithmen. Zum einem werden nicht vereinbare Einheiten miteinander in Verbindung gesetzt (Fehlerraten und Sekunden)
und zum anderen sind die in Abschnitt~\ref{sec:experimental} ermittelten Ma�e abh�ngig von dem ausgew�hlten Bildmaterial.
Ein Austausch des eingesetzten Materials kann dementsprechend zu einer differenzierenden \textit{Ranking}--Tabelle f�hren.
\begin{table}[H]
	\begin{center}
		\begin{tabular}{|l|c|c|c|c|}
			\mytoprule
			\centering\bfseries Eigenschaft & \bfseries \textit{RADISH} & \bfseries \textit{DCT} & \bfseries \textit{Wavelet} & \bfseries \textit{BMB}
			\\
			\hline
			\hline
			%---------------------------------------------------------------------------------------------------------------------	
			\textit{FRR} & 0,39 & 0,469 & \cellcolor{myorange}0,828 & \cellcolor{mygreen}0,256
			\\
			\hline
			%---------------------------------------------------------------------------------------------------------------------	
			\textit{FAR} & \cellcolor{myorange}0,028 & 0,0009 & \cellcolor{mygreen}0 & 0,02
			\\
			\hline	
			%---------------------------------------------------------------------------------------------------------------------	
			\textit{Geschwindigkeit} & 0,107s & 0,223s & \cellcolor{myorange}0,362s & \cellcolor{mygreen}0,038s 
			\\	
			\hline
			\hline
			%---------------------------------------------------------------------------------------------------------------------	
			\textit{Gesamt} & 0,175 & 0,23 & \cellcolor{myorange}0,39 & \cellcolor{mygreen}0,104 
			\\																						
			%---------------------------------------------------------------------------------------------------------------------
			\mybottomrule			
		\end{tabular}
		\caption{Ranking der Algorithmen (Eigene Darstellung)}
		\label{tab:conclusion_ranking}
	\end{center}
\end{table}
\noindent
Der Algorithmus mit den besten Werten bez�glich der betrachteten Eigenschaft wird mit der Farbe {\color{mygreen}\rule[0cm]{0.2cm}{0.2cm}}
hervorgehoben, wohingegen das Verfahren mit den schlechtesten Werten mit der Farbe {\color{myorange}\rule[0cm]{0.2cm}{0.2cm}} kodiert
wird.

\subsection{Zusammenfassung}
\label{sec:summary}
\begin{itemize}
	\item Alternativer Ansatz f�r die Pr�fung von Auth und Integrit�t: Perceptual Hashing
	\item Ausgw�hlte Algos sind der pHash Lib zu entnehmen gewesen
	\item Ansteuerung der Bibliothek erfolgte �ber ein eigens daf�r konzipiertes Interface (siehe Anhang)
	\item Analyse erfordert varrienden Bildmaterial (�hnlich, nicht �hnlich)
	\item F�r eine Analyse zwischen den Funktionweisen der Algos und des Bildmaterials wurden unterschiedliche Daten betrahctet (siehe Testdaten)
	\item Anschlie�end wurden Experiemnte zur Untersuchung von Erkennungsraten der Algos hinsichtlich des definierten Bildmaterials sowie
		  der Untersuchung der allgemeinen Geschwindkeit sowie die Steigerung zu nativen Ans�tzen
	\item Die protokollierten Ergebnisse wurden anschlie�end ausgewertet und mit anderen wissenschaftlichen Arbeiten in Bezug gesetzt.
	\item Die Erkenntnisse wurden in der in dem Abschnit XY zusammengefasst dargetsellt
\end{itemize}

\subsection{Ausblick}
\label{subsec:futureWork}
ToDo