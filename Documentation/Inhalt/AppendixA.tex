%#####################################################################################################################
% Datei	: AppendixA.tex
% Autor	: Byron Worms
%#####################################################################################################################
\section{Aufgabenbeschreibung}
\label{sec:appendix_a}
\boldtext{Tasks:}
\begin{enumerate}
	\item[\boldT (a)] \boldtext{Realization of an own setup of demonstrators}\\
	Install and test the pHash image perceptual hashing library.\\
	Familiarize yourself with the software and the basic principles implemented (see resources provided).	%----------------------------------------------------------------------------------------------------------------
	\item[\boldT (b)] \boldtext{Interplay with the CMR}\\
	Record / generate a set of image data with a wide variety of at least 100 different kinds of content to the course
	media repository (CMR). These image files should contain perceptually similar as well as dissimilar images. The
	resulting image repository is identified as CMR{I--Hashing}. Contribute to fulfil the media requirements of all
	other groups (incl. transfer of the usage right)!
	%----------------------------------------------------------------------------------------------------------------
	\item[\boldT (b)] \boldtext{Evaluation}\\
	Run the image perceptual hashing tool on the CMR subset(s) identified in section ''Course media repository (CMR)
	overview'' at the end of the document and perform a performance investigation.\\
	The evaluations have to include (but not limited to):
	\begin{itemize}[noitemsep]
		\item A analysis of achieved detection rates (especially on perceptually similar but not identical files)
		for the operational modes RADISH (radial hash), DCT hash and Marr/Mexican hat wavelet.
		\item A discussion of collision rates (dissimilar objects that are mapped onto the same perceptual hash)
		for the operation modes RADISH (radial hash), DCT hash and Marr/Mexican hat wavelet.
		\item A discussion of the improvement of search speeds by switching from searching whole image files to
		searching with perceptual hashes.
	\end{itemize}
	%----------------------------------------------------------------------------------------------------------------
	\item[\boldT (b)] \boldtext{Presentation}\\
	Documentation of the work performed in a (structured) layout (use the provided layout with max. 12 pages
	plus all requested appendixes) and presentation in lecture time (2x10 minutes presentation time for two-student
	team or 15 minutes for a one-student.	
\end{enumerate}