%#####################################################################################################################
% Datei	: AppendixA.tex
% Autor	: Byron Worms
%#####################################################################################################################
\section{Aufgabenbeschreibung}
\label{sec:appendix_a}
\boldtext{Tasks:}
\begin{enumerate}
	\item[\bold (a)] \boldtext{Realization of an own setup of demonstrators}\\
	Install and test the pHash image perceptual hashing library.\\
	Familiarize yourself with the software and the basic principles implemented (see resources provided).	%----------------------------------------------------------------------------------------------------------------
	\item[\bold (b)] \boldtext{Interplay with the CMR}\\
	Record / generate a set of image data with a wide variety of at least 100 different kinds of content to the course
	media repository (CMR). These image files should contain perceptually similar as well as dissimilar images. The
	resulting image repository is identified as CMR{I--Hashing}. Contribute to fulfil the media requirements of all
	other groups (incl. transfer of the usage right)!
	%----------------------------------------------------------------------------------------------------------------
	\item[\bold (b)] \boldtext{Evaluation}\\
	Run the image perceptual hashing tool on the CMR subset(s) identified in section ''Course media repository (CMR)
	overview'' at the end of the document and perform a performance investigation.\\
	The evaluations have to include (but not limited to):
	\begin{itemize}[noitemsep]
		\item A analysis of achieved detection rates (especially on perceptually similar but not identical files)
		for the operational modes RADISH (radial hash), DCT hash and Marr/Mexican hat wavelet.
		\item A discussion of collision rates (dissimilar objects that are mapped onto the same perceptual hash)
		for the operation modes RADISH (radial hash), DCT hash and Marr/Mexican hat wavelet.
		\item A discussion of the improvement of search speeds by switching from searching whole image files to
		searching with perceptual hashes.
	\end{itemize}
	%----------------------------------------------------------------------------------------------------------------
	\item[\bold (b)] \boldtext{Presentation}\\
	Documentation of the work performed in a (structured) layout (use the provided layout with max. 12 pages
	plus all requested appendixes) and presentation in lecture time (2x10 minutes presentation time for two-student
	team or 15 minutes for a one-student.	
\end{enumerate}
\begin{comment}
\begin{enumerate}
	\item[\bold (a)] \boldtext{Realization of an own setup of demonstrators}\\
	The task is to implement a keystroke dynamics (KD) acquisition and matching prototype from scratch (in JAVA).\\
	The tasks of the group are to implement an own demonstrator for this simple to acquire and process modality,
	using di-graph [Sim2007] and tri-graph representations of fixed-length input as feature space and use this
	prototype to perform user authentication.
	%----------------------------------------------------------------------------------------------------------------
	\item[\bold (b)] \boldtext{Data to be used}\\
	This group has to collect with their own tool KD data of the course participants. The set of samples must
	contain the following semantics:
	\begin{itemize}[noitemsep]
		\item[-] A pseudonym
		\item[-] A given pin ("77993")
		\item[-] A pin chosen by the donor of the samples
		\item[-] A sketch of a symbol chosen by the donor of the samples
		\item[-] The answer to the question "Where are you from?"
	\end{itemize}
	Furthermore, the participants in this group are asked to contribute to the data collections of the handwriting (HW)
	and (other) keystroke dynamic (KD) groups!
	%----------------------------------------------------------------------------------------------------------------	
	\item[\bold (c)] \boldtext{Evaluation}\\
	Run your prototype (see subtask (a) above) on the collected data (see subtask (b) above) and perform a performance
	evaluation with your prototype. The evaluation must include (but is not limited to):
	\begin{itemize}[noitemsep]
		\item An analysis of the authentication performance of the system using your own collected data
		\item An attack attempt using blind imposter attacks trying to force errors
		\item A projection of the samples in your data set to the characters of 'Doddingtons Zoo' [Doddington1998] and
		- if possible - an application of 'Doddingtons rules of thumbs' for the evaluation of the authentication performance
		of your demonstrator and data set
		\item A summary of the overall approach/algorithms/results/impact of the individual topic and a projection onto
		the biometric processing pipeline (see e.g. [Vielhauer2006])
	\end{itemize}
	%----------------------------------------------------------------------------------------------------------------	
	\item[\bold (d)] \boldtext{Presentation}\\
	Documentation of the work performed is done in a (structured) layout (ACM layout with max. 8 pages plus appendix)
	and presentation in lecture time (2x10 minutes presentation time for two-student team or 15 minutes for a one-student
	team).	
\end{enumerate}
\end{comment}