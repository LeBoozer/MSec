%#####################################################################################################################
% Datei	: Experimental.tex
% Autor	: Byron Worms
%#####################################################################################################################
%---------------------------------------------------------------------------------------------------------------------
% Experimental Investigation
%---------------------------------------------------------------------------------------------------------------------
\section{Experimentelle Untersuchungen}
\label{sec:experimental}
Die theoretische Definition und Beschreibung der einzelnen Algorithmen bieten keine M�glichkeit R�ckschl�sse �ber die generelle
Einsetzbarkeit in Abh�ngigkeit des verwendeten Bildmaterials zu ziehen. Aufgrund der hohen Komplexit�t und Varianz in
den bildlichen Strukturen, k�nnen keine allgemeing�ltigen Aussagen �ber den Zusammenhang zwischen fehlerhaften Ergebnissen
der Verfahren und dem strukturellen Aufbau der Bilder getroffen werden.
\\
Mit Hilfe der in Sektion~\ref{sec:solution} (\textit{L�sungsansatz}) wohldefinierten Kategorien an variierenden
Bildmaterials, in Bezug auf Komplexit�t sowie Strukturaufbau, soll eine potentielle Korrelation zwischen den strukturellen
Gegebenheiten und den grundlegenden Funktionsweisen der Algorithmen untersucht werden.
\\
Bereits in der Wissenschaft bekannte Fehleranf�lligkeiten der genutzten Strategien werden dabei als solche markiert und nicht
n�her erl�utert.
\\[1em]
Die nachfolgende Auflistung beinhaltet eine �bersicht der durchgef�hrten und in Kategorien gruppierten Experimente:
\begin{enumerate}[noitemsep]
	\item \boldtext{Fundamentale Funktionalit�tstests}
		\begin{itemize}[noitemsep]
			\item Erkennungsraten bei Identit�tstests
			\item Erkennungsraten der \textit{Einfachen Farben}
		\end{itemize}
		
	\item \boldtext{tbd}
		\begin{itemize}[noitemsep]
			\item Originalbild vs. Modifikationen (\textit{Elementare Formen})
			\item Originalbild vs. Modifikationen (\textit{Komplexe Bilder})
			\item Erkennungsraten beim Quervergleich \textit{Komplexe Bilder}
		\end{itemize}		
		
	\item \boldtext{Geschwindigkeitsanalysen}
		\begin{itemize}[noitemsep]
			\item Berechnungszeiten der Algorithmen
			\item tbd (Hash vs Ganzbild)
		\end{itemize}
\end{enumerate}
\noindent
TO DO: Fehlerraten oder so erw�hnen (JEWEILS GEMESSENDE DATEN)\\
\boldtext{Testsystem und Parametrisierung}\\
F�r die Gew�hrleistung, dass die einzelnen Experimente in weitestgehend system-- sowie hardwareunabh�ngigen Ergebnissen resultieren,
geschieht deren Durchf�hrung ausschlie�lich auf dem nachstehenden System:
\begin{table}[H]
	\begin{center}
		\begin{tabular}{|l|l|}
			\mytoprule
			%---------------------------------------------------------------------------------------------------------------------	
			Betriebssystem & Windows 8.1, x64
			\\
			\hline
			%---------------------------------------------------------------------------------------------------------------------	
			Prozessor & AMD x6 1055t (~3.3GHz)
			\\
			\hline	
			%---------------------------------------------------------------------------------------------------------------------	
			RAM & 8GB DDR3
			\\
			\hline	
			%---------------------------------------------------------------------------------------------------------------------	
			Festplatte & SamsungSSD 850 EVO 512GB
			\\																			
			%---------------------------------------------------------------------------------------------------------------------
			\mybottomrule			
		\end{tabular}
		\caption{Testsystem (eigene Darstellung)}
		\label{tab:experimental_testsystem}
	\end{center}
\end{table}	
\noindent
Die Parametrisierung der unterschiedlichen Algorithmen erfolgt nach den in der Bibliothek \textit{pHash} angegebenen Standardwerte:
\begin{table}[H]
	\begin{center}
		\begin{tabular}{|l|l|}
			\mytoprule
			\centering\bfseries Algorithmus & \bfseries Parameter(--typ)
			\\
			\hline
			\hline
			%---------------------------------------------------------------------------------------------------------------------	
			RADISH & Gamma: 1 \\
			& Sigma: 2 \\
			& \#Winkel: 180
			\\
			\hline
			%---------------------------------------------------------------------------------------------------------------------	
			DCT & --
			\\
			\hline	
			%---------------------------------------------------------------------------------------------------------------------	
			Wavelet & Alpha: 2 \\
			& Level: 1
			\\
			\hline	
			%---------------------------------------------------------------------------------------------------------------------	
			BMB & Methode: 1
			\\																			
			%---------------------------------------------------------------------------------------------------------------------
			\mybottomrule			
		\end{tabular}
		\caption{Parametrisierung der Algorithmen (nach~\cite{PHASH})}
		\label{tab:experimental_params}
	\end{center}
\end{table}	
\noindent

\subsection{Fundamentale Funktionalit�tstests}
\subsubsection{Erkennungsraten bei Identit�tstests}
Die Durchf�hrung des Identit�tstests dient als erforderliche Grundlage f�r alle nachfolgenden aufgestellten Testf�lle und wird
daher f�r alle vier Algorithmen realisiert. W�hrend der Ausf�hrung werden die Bilder aus den Kategorien \textit{Elementare Formen}
sowie \textit{Komplexe Bilder} gegen die eigene Identit�t verglichen und die eventuell entstehenden Fehler protokolliert.
\\
Das Experiment umfasste insgesamt 288 unterschiedliche Bilddaten und f�hrte zu dem nachstehenden Ergebnis:
\begin{equation*}
	FRR = 0
\end{equation*}

\subsubsection{Erkennungsraten der \textit{Einfachen Farben}}
Das Experiment hat einen Quervergleich aller 9 einfarbigen Bilder der Kategorie \textit{Einfache Farben} mit allen vier Verfahren durchgef�hrt.
Die entstehenden Resultate basieren dabei auf der Natur der internen Funktionsweise der Vergleichsalgorithmen f�r die Hashdaten
(detaillierte Erl�uterung kann TODO TODO TODO).
\\
Der Test beinhaltete 45 Vergleichsm�glichkeiten und generierte das folgende Ergebnis:
\begin{align*}
	FRR = 1 \\
	FAR = 1
\end{align*}

\subsection{tbd}
\subsubsection{Originalbild vs. Modifikationen (\textit{Elementare Formen}}
\subsubsection{Originalbild vs. Modifikationen (\textit{Komplexe Bilder}}
\subsubsection{Erkennungsraten beim Quervergleich \textit{Komplexe Bilder}}

\subsection{Geschwindigkeitsanalysen}
\subsubsection{Berechnungszeiten der Algorithmen}
\subsubsection{tbd}